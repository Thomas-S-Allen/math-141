\documentclass[10pt]{article}\usepackage[]{graphicx}\usepackage[]{color}
%% maxwidth is the original width if it is less than linewidth
%% otherwise use linewidth (to make sure the graphics do not exceed the margin)
\makeatletter
\def\maxwidth{ %
  \ifdim\Gin@nat@width>\linewidth
    \linewidth
  \else
    \Gin@nat@width
  \fi
}
\makeatother

\definecolor{fgcolor}{rgb}{0.345, 0.345, 0.345}
\newcommand{\hlnum}[1]{\textcolor[rgb]{0.686,0.059,0.569}{#1}}%
\newcommand{\hlstr}[1]{\textcolor[rgb]{0.192,0.494,0.8}{#1}}%
\newcommand{\hlcom}[1]{\textcolor[rgb]{0.678,0.584,0.686}{\textit{#1}}}%
\newcommand{\hlopt}[1]{\textcolor[rgb]{0,0,0}{#1}}%
\newcommand{\hlstd}[1]{\textcolor[rgb]{0.345,0.345,0.345}{#1}}%
\newcommand{\hlkwa}[1]{\textcolor[rgb]{0.161,0.373,0.58}{\textbf{#1}}}%
\newcommand{\hlkwb}[1]{\textcolor[rgb]{0.69,0.353,0.396}{#1}}%
\newcommand{\hlkwc}[1]{\textcolor[rgb]{0.333,0.667,0.333}{#1}}%
\newcommand{\hlkwd}[1]{\textcolor[rgb]{0.737,0.353,0.396}{\textbf{#1}}}%
\let\hlipl\hlkwb

\usepackage{framed}
\makeatletter
\newenvironment{kframe}{%
 \def\at@end@of@kframe{}%
 \ifinner\ifhmode%
  \def\at@end@of@kframe{\end{minipage}}%
  \begin{minipage}{\columnwidth}%
 \fi\fi%
 \def\FrameCommand##1{\hskip\@totalleftmargin \hskip-\fboxsep
 \colorbox{shadecolor}{##1}\hskip-\fboxsep
     % There is no \\@totalrightmargin, so:
     \hskip-\linewidth \hskip-\@totalleftmargin \hskip\columnwidth}%
 \MakeFramed {\advance\hsize-\width
   \@totalleftmargin\z@ \linewidth\hsize
   \@setminipage}}%
 {\par\unskip\endMakeFramed%
 \at@end@of@kframe}
\makeatother

\definecolor{shadecolor}{rgb}{.97, .97, .97}
\definecolor{messagecolor}{rgb}{0, 0, 0}
\definecolor{warningcolor}{rgb}{1, 0, 1}
\definecolor{errorcolor}{rgb}{1, 0, 0}
\newenvironment{knitrout}{}{} % an empty environment to be redefined in TeX

\usepackage{alltt}

\usepackage{amsmath,amssymb,amsthm}
\usepackage{fancyhdr,url,hyperref}
\usepackage{enumerate,multirow}

\oddsidemargin 0in  %0.5in
\topmargin     0in
\leftmargin    0in
\rightmargin   0in
\textheight    9in
\textwidth     6in %6in

\pagestyle{fancy}

\lhead{\textsc{MATH 141}}
\chead{\textsc{Problem Set 3}}
\lfoot{}
\cfoot{}
%\cfoot{\thepage}
\rfoot{}
\renewcommand{\headrulewidth}{0.2pt}
\renewcommand{\footrulewidth}{0.0pt}

\newcommand{\ans}{\vspace{0.25in}}
\newcommand{\R}{{\sf R}\xspace}
\newcommand{\cmd}[1]{\texttt{#1}}

\title{MATH 141:\\Intro to Probability and Statistics}
\date{Fall 2017}

\rhead{\textsc{Fall 2017}}
\IfFileExists{upquote.sty}{\usepackage{upquote}}{}
\begin{document}
%\SweaveOpts{concordance=TRUE}

\begin{enumerate}
\item The American Community 
Survey is an ongoing survey that provides data every year to give communities the 
current information they need to plan investments and services. The 2010 American 
Community Survey estimates that 14.6\% of Americans live below the poverty line, 
20.7\% speak a language other than English (foreign language) at home, and 4.2\% 
fall into both categories.
\begin{enumerate}[(a)]
\item Are living below the poverty line and speaking a foreign language at home 
disjoint?
\item Draw a Venn diagram summarizing the variables and their associated 
probabilities.
\item What percent of Americans live below the poverty line and only speak 
English at home?
\item What percent of Americans live below the poverty line or speak a foreign 
language at home?
\item What percent of Americans live above the poverty line and only speak 
English at home? 
\item Is the event that someone lives below the poverty line independent of the 
event that the person speaks a foreign language at home?
\end{enumerate}


\item In parts~(a) and~(b), 
identify whether the events are disjoint, independent, or neither (events cannot 
be both disjoint and independent).
\begin{enumerate}[(a)]
\item You and a randomly selected student from your class both earn A's in this 
course. 
\item You and your class study partner both earn A's in this course.
\item If two events can occur at the same time, must they be dependent?
\end{enumerate}


\item Data collected at elementary 
schools in DeKalb County, GA suggest that each year roughly 25\% of students miss 
exactly one day of school, 15\% miss 2 days, and 28\% miss 3 or more days due to 
sickness.
\begin{enumerate}[(a)]
\item What is the probability that a student chosen at random doesn't miss any 
days of school due to sickness this year?
\item What is the probability that a student chosen at random misses no more than 
one day?
\item What is the probability that a student chosen at random misses at least one 
day?
\item If a parent has two kids at a DeKalb County elementary school, what is the 
probability that neither kid will miss any school? Note any assumption you must 
make to answer this question.
\item If a parent has two kids at a DeKalb County elementary school, what is the 
probability that both kids will miss some school, i.e. at least one day? Note any 
assumption you make.
\item If you made an assumption in part~(d) or~(e), do you think it was 
reasonable? If you didn't make any assumptions, double check your earlier answers.
\end{enumerate}

\item $\mathbb{P}(A) = 0.3$, 
$\mathbb{P}(B) = 0.7$
\begin{enumerate}[(a)]
\item Can you compute $\mathbb{P}(A \text{ and } B)$ if you only know $\mathbb{P}(A)$ and $\mathbb{P}(B)$?
\item Assuming that events $A$ and $B$ arise from independent random processes,
\begin{enumerate}[(i)]
\item what is $\mathbb{P}(A \text{ and } B)$?
\item what is $\mathbb{P}(A \text{ or } B)$?
\item what is $\mathbb{P}(A \,|\, B)$?
\end{enumerate}
\item If we are given that $\mathbb{P}(A \text{ and } B) = 0.1$, are the random variables giving rise 
to events $A$ and $B$ independent?
\item If we are given that $\mathbb{P}(A \text{ and } B) = 0.1$, what is $\mathbb{P}(A \,|\, B)$?
\end{enumerate}

\item A 2010 Pew Research poll asked 
1,306 Americans ``From what you've read and heard, is there solid evidence that 
the average temperature on earth has been getting warmer over the past few 
decades, or not?". The table below shows the distribution of responses by party 
and ideology, where the counts have been replaced with relative frequencies.
\begin{center}
\begin{tabular}{ll  ccc c} 
                    &                           & \multicolumn{3}{c}{\textit{Response}} \\
\cline{3-5}
                    &                           & Earth is  & Not       & Don't Know    &   \\
                    &                           & warming   & warming   & Refuse        & Total\\
\cline{2-6}
                    & Conservative Republican   & 0.11      & 0.20      & 0.02      & 0.33  \\
\textit{Party and}  & Mod/Lib Republican        & 0.06      & 0.06      & 0.01      & 0.13 \\
\textit{Ideology}   & Mod/Cons Democrat         & 0.25      & 0.07      & 0.02      & 0.34 \\
                    & Liberal Democrat          & 0.18      & 0.01      & 0.01      & 0.20\\
\cline{2-6}
                    &Total                      & 0.60      & 0.34      & 0.06      & 1.00
\end{tabular}
\end{center}
\begin{enumerate}[(a)]
\item Are believing that the earth is warming and being a liberal Democrat mutually 
exclusive?
\item What is the probability that a randomly chosen respondent believes the 
earth is warming or is a liberal Democrat?
\item What is the probability that a randomly chosen respondent believes the 
earth is warming given that he is a liberal Democrat?
\item What is the probability that a randomly chosen respondent believes the 
earth is warming given that he is a conservative Republican?
\item Does it appear that whether or not a respondent believes the earth is 
warming is independent of their party and ideology? Explain your reasoning.
\item What is the probability that a randomly chosen respondent is a 
moderate/liberal Republican given that he does not believe that the earth is 
warming? 
\end{enumerate}

\item After an introductory 
statistics course, 80\% of students can successfully construct box plots. Of 
those who can construct box plots, 86\% passed, while only 65\% of those students 
who could not construct box plots passed.
\begin{enumerate}[(a)]
\item Construct a tree diagram of this scenario.
\item Calculate the probability that a student is able to construct a box plot 
if it is known that he passed.
\end{enumerate}


\item Lupus is a medical phenomenon where 
antibodies that are supposed to attack foreign cells to prevent infections 
instead see plasma proteins as foreign bodies, leading to a high risk of blood 
clotting. It is believed that 2\% of the population suffer from this disease. The 
test is 98\% accurate if a person actually has the disease. The test is 74\% 
accurate if a person does not have the disease. There is a line from the Fox 
television show \emph{House} that is often used after a patient tests positive 
for lupus: ``It's never lupus." Do you think there is truth to this statement? 
Use appropriate probabilities to support your answer.


\end{enumerate}

\textbf{Extra Challenge Problem:} A \emph{chord} of a circle is a straight line 
segment whose endpoints both lie on the circle. For a fixed circle, what is
the probability that the length of a randomly drawn chord will exceed that
circle's radius?

\end{document}
